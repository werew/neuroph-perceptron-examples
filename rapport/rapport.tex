\documentclass[twoside,openright,a4paper,11pt,french]{article}
\usepackage[utf8]{inputenc}
\usepackage[french]{babel}
\usepackage[T1]{fontenc}
\usepackage{emptypage}
\usepackage{amsmath}

% Utilisation d'url
\usepackage{url}
\urlstyle{sf}

% Utilisation d'images, stockées dans le répertoire ./pics/
\usepackage{graphicx}
\graphicspath{pics/}

% Définition des marges
\usepackage{geometry}
\geometry{
  left=25mm,
  right=25mm,
  top=25mm,
  bottom=25mm,
  foot=15mm
}
\usepackage{listings}
\usepackage{color}

\definecolor{gray}{rgb}{0.8,0.8,0.8}

\begin{document}

\pagestyle{plain}
\setlength{\parindent}{0pt}
% La page de garde
\thispagestyle{empty}

\begin{center}
       \noindent
       \includegraphics[height=2.5cm]{./pics/uds.eps}       
       
       \vfill\vfill

    {\large \textsc{Licence 3 de Sciences, mention Informatique}}

    \bigskip\bigskip

    {\large \textsc{Intelligence Artificielle}}

    \vfill\vfill

% Titre du document
    {\huge \sc
      \begin{center} 
        Rapport sur le projet: \\
        Perceptron et perceptron multi-couchers avec Neuroph
      \end{center}}

    \vfill\vfill

    {\large Présenté par}

\medskip

% Identité des auteurs
    {\large Victor \textsc{Constans}}\\
    {\large Luigi  \textsc{Coniglio}}\\
\bigskip

\end{center}



% La table des matières
\parskip=0pt
\tableofcontents


\vspace{5cm}

%Start content

\section{Fonctions booléennes}

Dans cette premiere section de ce rapport on se propose
de implementer des fonctions booléennes a l'aide des reseaux
de neurones. 

\subsection{"Et" logique}

En matematique la conjonction logique $\land$ est un
connecteur logique que, etant donne deux propositions $A$ et $B$
forme une nouvelle propositions $A \land B$ qui est vrai seulement
si $A$ at $B$ sont vraies.

\begin{table}[h]
  \centering
% On paramètre ici le placement du texte dans les cases, en mode paragraphe de 5cm de large dans la case de gauche ("p{5cm}") et automatique avec un alignement à droite dans la case de droite ('r')
  \begin{tabular}{| c | c | c |}
    \hline
    \textbf{$A$} & \textbf{$B$} & \textbf{$A \land B$}\\
    \hline
    0 & 0  & 0 \\
    \hline
    0 & 1  & 0 \\
    \hline
    1 & 0  & 0 \\
    \hline
    1 & 1  & 1 \\
    \hline
  \end{tabular}
  \caption{Table de verite de $A \land B$}
  \label{tab:et}
\end{table}







%End content

\addcontentsline{toc}{section}{Références}
\bibliographystyle{plain}
\bibliography{rapport}

\end{document}
